%===============================================================
%Template for UPC-TALP based on CTU and modified to match UPC-TALP colors and style  
%The original template from Czech Technical University  % Author: Martin Malý.
% They are defined by new graphical manual - 2017.
% Share and modify as you like. Keep the name of the authors.
% It is forbidden to use the template commercially.
%===============================================================

\documentclass{beamer}
\usepackage[utf8]{inputenc}
\usepackage{comment}
\usetheme{Madrid}
  \usepackage[maxbibnames=99]{biblatex}
\definecolor{cvut_navy}{HTML}{0065BD}
\definecolor{cvut_blue}{HTML}{6AADE4}
\definecolor{cvut_gray}{HTML}{156570}
\usepackage{biblatex}
\usepackage{tikz}
\usepackage{pgfplots}
\usepackage[makeroom]{cancel}
\usepackage{threeparttable}
\usepackage[utf8]{inputenc}
\usepackage[usenames,dvipsnames]{xcolor}
\addbibresource{biblatex-examples.bib}
\renewcommand*{\familydefault}{\sfdefault}
\setbeamercolor{section in toc}{}
\setbeamercolor{section in toc}{fg=black,bg=yellow} 
\setbeamercolor{alerted text}{fg=cvut_blue}
\usepackage{tikzsymbols}
\usepackage{textcomp}
\usepackage{parskip}
\definecolor{darkblue}{rgb}{0, 0, 0.5} 
\definecolor{babyblue}{rgb}{0.54, 0.81, 0.94}
\usepackage{pgf}
\usepackage{color,soul}
\usepackage{pythontex} 
\usepackage{tcolorbox}
\tcbuselibrary{skins}
\usepackage{minted}
\usepackage{xcolor,soul}
\definecolor{lightblue}{rgb}{.90,.95,1}
\sethlcolor{lightblue}
\renewcommand<>{\hl}[1]{\only#2{\beameroriginal{\hl}}{#1}}
\setbeamertemplate{page number in head/foot}[framenumber]
%%% attravive box over equestion
\usepackage{empheq}
\usepackage{xcolor}
\definecolor{lightgreen}{HTML}{90EE90}
\newcommand{\boxedeq}[2]{\begin{empheq}[box={\fboxsep=6pt\fbox}]{align}\label{#1}#2\end{empheq}}
\newcommand{\coloredeq}[2]{\begin{empheq}[box=\colorbox{lightgreen}]{align}\label{#1}#2\end{empheq}}
\newcommand{\highlight}[1]{%
  \colorbox{red!40}{$\displaystyle#1$}}
  
  
  \definecolor{babyblue}{rgb}{0.54, 0.81, 0.94}
\definecolor{babypink}{rgb}{0.96, 0.76, 0.76}
\definecolor{blue(ncs)}{rgb}{0.0, 0.53, 0.74}
\definecolor{pistachio}{rgb}{0.58, 0.77, 0.45}
\definecolor{darksalmon}{rgb}{0.91, 0.59, 0.48}
\definecolor{lightsalmonpink}{rgb}{1.0, 0.6, 0.6}
\definecolor{columbiablue}{rgb}{0.61, 0.87, 1.0}
\definecolor{corn}{rgb}{0.98, 0.93, 0.36}
\definecolor{jonquil}{rgb}{0.98, 0.85, 0.37}
\definecolor{bananayellow}{rgb}{1.0, 0.88, 0.21}
\newcommand{\bert}{\ensuremath{%
  \mathchoice{\includegraphics[height=2ex]{Bert-pic-removebg-preview.png}} 
    {\includegraphics[height=2ex]{Bert-pic-removebg-preview.png}}
    {\includegraphics[height=1.5ex]{Bert-pic-removebg-preview.png}}
    {\includegraphics[height=1ex]{Bert-pic-removebg-preview.png}}
}}
  
  
\useoutertheme{infolines}



\usepackage{courier}
%\usepackage{animate}  
\usepackage{expl3}
\usepackage[listings,theorems]{tcolorbox}



%%%%%%%%%%%%%%%%%%%%%%%%%%%%%%%%%%%%%%%%%%%%%%%%%%%%%%%%%%%%%%%%%%%%%  
%commands for simulating terminal in/output  
%\scroll[<line separator string>]{<width as TeX dim>} 
%                             {<number of lines>}{terminal text line}  
%\clearbuf  %clears line buffer  
%%%%%%%%%%%%%%%%%%%%%%%%%%%%%%%%%%%%%%%%%%%%%%%%%%%%%%%%%%%%%%%%%%%%%  
\ExplSyntaxOn
\seq_new:N\g_linebuffer_seq
\seq_new:N\g_inputline_seq



\newcommand\scroll[4][§§]{
  \seq_set_split:Nnn\g_inputline_seq{#1}{#4}
  \seq_map_inline:Nn\g_inputline_seq{
    \seq_gput_right:Nx\g_linebuffer_seq{##1}
    \int_compare:nT{\seq_count:N\g_linebuffer_seq>#3}{
      \seq_gpop_left:NN\g_linebuffer_seq\dummy
    }
  }
  \fbox{\begin{minipage}[t][#3\baselineskip]{#2}
    \ttfamily
    \seq_map_inline:Nn\g_linebuffer_seq{\mbox{##1}\\}
  \end{minipage}}
}
\newcommand\clearbuf{\seq_gclear:N\g_linebuffer_seq}
\ExplSyntaxOff

\setbeamertemplate{headline}{%
\begin{beamercolorbox}[colsep=1.5pt]{upper separation line head}
\end{beamercolorbox}
\begin{beamercolorbox}{section in head/foot}
    \vskip2pt\insertsectionnavigationhorizontal{\paperwidth}{}{\hskip0pt plus1filll}\vskip2pt
\end{beamercolorbox}%
%\begin{beamercolorbox}[ht=10pt]{subsection in head/foot}%
%    \vskip2pt\insertsubsectionnavigationhorizontal{\paperwidth}{}{\hskip0pt plus1filll}\vskip2pt
%\end{beamercolorbox}%
\begin{beamercolorbox}[colsep=1.5pt]{lower separation line head}
\end{beamercolorbox}
}
\makeatletter
\newcommand\SoulColor{%
  \let\set@color\beamerorig@set@color
  \let\reset@color\beamerorig@reset@color}
\makeatother
\SoulColor
\usepackage{amsmath, bm}
\usepackage{tikz}
\setbeamercovered{dynamic}

\newcommand{\highlightt}[1]{%
  \colorbox{blue!40}{$\displaystyle#1$}}


\newenvironment<>{problock}[1]{%
  \begin{actionenv}#2%
      \def\insertblocktitle{#1}%
      \par%
      \mode<presentation>{%
       % \setbeamercolor{block title}{fg=white,bg=orange!20!black}
        %\setbeamercolor{block title}{fg=white,bg=red!10!black}
        \setbeamercolor{block title}{fg=white,bg=cvut_blue}
       \setbeamercolor{block body}{fg=black,bg=white!50}
       \setbeamercolor{itemize item}{fg=orange!20!black}
       \setbeamertemplate{itemize item}[triangle]
     }%
      \usebeamertemplate{block begin}
    \par\usebeamertemplate{block end}
    \end{actionenv}
    }
    
    \newcommand<>{\uncovergraphics}[2][{}]{
    % Taken from: <https://tex.stackexchange.com/a/354033/95423>
    \begin{tikzpicture}
    \node[anchor=south west,inner sep=0] (B) at (4,0)
        {\includegraphics[#1]{#2}};
    \alt#3{}{%
        \fill [draw=none, fill=background, fill opacity=0.9] (B.north west) -- (B.north east) -- (B.south east) -- (B.south west) -- (B.north west) -- cycle;
    }
    \end{tikzpicture}
}




\newcommand{\FourQuads}[4]{
    \begin{columns}[onlytextwidth]
        \begin{column}{.45\textwidth}
            \begin{block}{\small Dataset and Task}
            %\begin{tcolorbox}[enhanced,drop shadow, title=Example with tcolorbox]
            %\begin{tcolorbox}[enhanced,colframe=white,colback=white, fuzzy halo = 1mm with gray] 
            %\begin{tcolorbox}[colback=yellow!20,colframe=yellow,title=Dataset and Task]
                \begin{minipage}[t][.25\textheight][t]{\textwidth}
                    #1
                \end{minipage}
               % \end{tcolorbox}
            \end{block}
        \end{column}
        \begin{column}{.45\textwidth}
            \begin{block}{\small Word-level based Visual Re-ranker}
                \begin{minipage}[t][.25\textheight][t]{\textwidth}
                    #2
                \end{minipage}
            \end{block}
        \end{column}        
    \end{columns}
    \begin{columns}[onlytextwidth]
        \begin{column}{.45\textwidth}
            \begin{block}{\small Sentence based Visual Re-ranker}
                \begin{minipage}[t][.25\textheight][t]{\textwidth}
                    #3
                \end{minipage}
            \end{block}
        \end{column}
        \begin{column}{.45\textwidth}
            \begin{block}{\small More Task - Caption generation }
                \begin{minipage}[t][.25\textheight][t]{\textwidth}
                    #4
                \end{minipage}
            \end{block}
        \end{column}        
    \end{columns}   
}    

\newlength{\overwritelength}
\newlength{\minimumoverwritelength}
\setlength{\minimumoverwritelength}{1cm}
\newcommand{\overwrite}[3][red]{%
  \settowidth{\overwritelength}{$#2$}%
  \ifdim\overwritelength<\minimumoverwritelength%
    \setlength{\overwritelength}{\minimumoverwritelength}\fi%
  \stackrel
    {%
      \begin{minipage}{\overwritelength}%
        \color{#1}\centering\small #3\\%
        \rule{1pt}{9pt}%
      \end{minipage}}
    {\colorbox{#1!50}{\color{black}$\displaystyle#2$}}}


\newlength{\overwritelength}
\newlength{\minimumoverwritelength}
\setlength{\minimumoverwritelength}{1cm}
\newcommand{\overwritee}[3][blue]{%
  \settowidth{\overwritelength}{$#2$}%
  \ifdim\overwritelength<\minimumoverwritelength%
    \setlength{\overwritelength}{\minimumoverwritelength}\fi%
  \stackrel
    {%
      \begin{minipage}{\overwritelength}%
        \color{#1}\centering\small #3\\%
        \rule{1pt}{9pt}%
      \end{minipage}}
    {\colorbox{#1!50}{\color{black}$\displaystyle#2$}}}



\newlength{\overwritelength}
\newlength{\minimumoverwritelength}
\setlength{\minimumoverwritelength}{1cm}
\newcommand{\overwriteee}[3][gray]{%
  \settowidth{\overwritelength}{$#2$}%
  \ifdim\overwritelength<\minimumoverwritelength%
    \setlength{\overwritelength}{\minimumoverwritelength}\fi%
  \stackrel
    {%
      \begin{minipage}{\overwritelength}%
        \color{#1}\centering\small #3\\%
        \rule{1pt}{9pt}%
      \end{minipage}}
    {\colorbox{#1!50}{\color{black}$\displaystyle#2$}}}



%%%%%%%%%%%%%%%% selected Dataset and Task 

\newcommand{\FourQuadss}[4]{
    \begin{columns}[onlytextwidth]
        \begin{column}{.45\textwidth}
            \begin{alertblock}{\small Dataset and Task}
            %\begin{tcolorbox}[enhanced,drop shadow, title=Example with tcolorbox]
            %\begin{tcolorbox}[enhanced,colframe=white,colback=white, fuzzy halo = 1mm with gray] 
            %\begin{tcolorbox}[colback=yellow!20,colframe=yellow,title=Dataset and Task]
                \begin{minipage}[t][.25\textheight][t]{\textwidth}
                    #1
                \end{minipage}
               % \end{tcolorbox}
            \end{alertblock}
        \end{column}
        \begin{column}{.45\textwidth}
            \begin{block}{\small Word-level based Visual Re-ranker}
                \begin{minipage}[t][.25\textheight][t]{\textwidth}
                    #2
                \end{minipage}
            \end{block}
        \end{column}        
    \end{columns}
    \begin{columns}[onlytextwidth]
        \begin{column}{.45\textwidth}
            \begin{block}{\small Sentence based Visual Re-ranker}
                \begin{minipage}[t][.25\textheight][t]{\textwidth}
                    #3
                \end{minipage}
            \end{block}
        \end{column}
        \begin{column}{.45\textwidth}
            \begin{block}{\small More Task - Caption generation}
                \begin{minipage}[t][.25\textheight][t]{\textwidth}
                    #4
                \end{minipage}
            \end{block}
        \end{column}        
    \end{columns}   
}    



\newcommand{\FourQuadsss}[4]{
    \begin{columns}[onlytextwidth]
        \begin{column}{.45\textwidth}
            \begin{alertblock}{\small Dataset and Task}
            %\begin{tcolorbox}[enhanced,drop shadow, title=Example with tcolorbox]
            %\begin{tcolorbox}[enhanced,colframe=white,colback=white, fuzzy halo = 1mm with gray] 
            %\begin{tcolorbox}[colback=yellow!20,colframe=yellow,title=Dataset and Task]
                \begin{minipage}[t][.25\textheight][t]{\textwidth}
                    #1
                \end{minipage}
               % \end{tcolorbox}
            \end{alertblock}
        \end{column}
        \begin{column}{.45\textwidth}
            \begin{alertblock}{\small Word-level based Visual Re-ranker}
                \begin{minipage}[t][.25\textheight][t]{\textwidth}
                    #2
                \end{minipage}
            \end{alertblock}
        \end{column}        
    \end{columns}
    \begin{columns}[onlytextwidth]
        \begin{column}{.45\textwidth}
            \begin{block}{\small Sentence based Visual Re-ranker}
                \begin{minipage}[t][.25\textheight][t]{\textwidth}
                    #3
                \end{minipage}
            \end{block}
        \end{column}
        \begin{column}{.45\textwidth}
            \begin{block}{\small More Task - Caption generation}
                \begin{minipage}[t][.25\textheight][t]{\textwidth}
                    #4
                \end{minipage}
            \end{block}
        \end{column}        
    \end{columns}   
}    


\newcommand{\FourQuadssss}[4]{
    \begin{columns}[onlytextwidth]
        \begin{column}{.45\textwidth}
            \begin{alertblock}{\small Dataset and Task}
            %\begin{tcolorbox}[enhanced,drop shadow, title=Example with tcolorbox]
            %\begin{tcolorbox}[enhanced,colframe=white,colback=white, fuzzy halo = 1mm with gray] 
            %\begin{tcolorbox}[colback=yellow!20,colframe=yellow,title=Dataset and Task]
                \begin{minipage}[t][.25\textheight][t]{\textwidth}
                    #1
                \end{minipage}
               % \end{tcolorbox}
            \end{alertblock}
        \end{column}
        \begin{column}{.45\textwidth}
            \begin{alertblock}{\small Word-level based Visual Re-ranker}
                \begin{minipage}[t][.25\textheight][t]{\textwidth}
                    #2
                \end{minipage}
            \end{alertblock}
        \end{column}        
    \end{columns}
    \begin{columns}[onlytextwidth]
        \begin{column}{.45\textwidth}
            \begin{alertblock}{\small Sentence based Visual Re-ranker}
                \begin{minipage}[t][.25\textheight][t]{\textwidth}
                    #3
                \end{minipage}
            \end{alertblock}
        \end{column}
        \begin{column}{.45\textwidth}
            \begin{block}{\small More Task - Caption generation}
                \begin{minipage}[t][.25\textheight][t]{\textwidth}
                    #4
                \end{minipage}
            \end{block}
        \end{column}        
    \end{columns}   
}    


\newcommand{\FourQuadsssss}[4]{
    \begin{columns}[onlytextwidth]
        \begin{column}{.45\textwidth}
            \begin{alertblock}{\small Dataset and Task}
            %\begin{tcolorbox}[enhanced,drop shadow, title=Example with tcolorbox]
            %\begin{tcolorbox}[enhanced,colframe=white,colback=white, fuzzy halo = 1mm with gray] 
            %\begin{tcolorbox}[colback=yellow!20,colframe=yellow,title=Dataset and Task]
                \begin{minipage}[t][.25\textheight][t]{\textwidth}
                    #1
                \end{minipage}
               % \end{tcolorbox}
            \end{alertblock}
        \end{column}
        \begin{column}{.45\textwidth}
            \begin{alertblock}{\small Word-level based Visual Re-ranker}
                \begin{minipage}[t][.25\textheight][t]{\textwidth}
                    #2
                \end{minipage}
            \end{alertblock}
        \end{column}        
    \end{columns}
    \begin{columns}[onlytextwidth]
        \begin{column}{.45\textwidth}
            \begin{alertblock}{\small Sentence based Visual Re-ranker}
                \begin{minipage}[t][.25\textheight][t]{\textwidth}
                    #3
                \end{minipage}
            \end{alertblock}
        \end{column}
        \begin{column}{.45\textwidth}
            \begin{alertblock}{\small More Task - Caption generation}
                \begin{minipage}[t][.25\textheight][t]{\textwidth}
                    #4
                \end{minipage}
            \end{alertblock}
        \end{column}        
    \end{columns}   
}    



%\addbibresource{biblatex-examples.bib}
    
%\setbeamercolor{block body}{fg=black, bg=green!20!white}
%\setbeamercolor{block title}{fg=black, bg=green!60!black}

%\BeforeBeginEnvironment{definition}{%
%    \setbeamercolor{block title}{fg=black, bg=green!20!white}
% %   \setbeamercolor{block body}{fg=black, bg=green!60!black}
%}
%\AfterEndEnvironment{definition}{
% \setbeamercolor{block title}{use=structure,fg=structure.fg,bg=structure.fg!20!bg}
% \setbeamercolor{block body}{parent=normal text,use=block title,bg=block title.bg!50!bg, fg=black}
%}

\newlength\dlf
\newcommand\alignedbox[3][yellow]{
  % #1 = color (optional, defaults to yellow)
  % #2 = before alignment
  % #3 = after alignment
  &
  \begingroup
  \settowidth\dlf{$\displaystyle #2$}
  \addtolength\dlf{\fboxsep+\fboxrule}
  \hspace{-\dlf}
  \fcolorbox{red}{#1}{$\displaystyle #2 #3$}
  \endgroup
}

\usepackage{collcell}
\usepackage{booktabs}
\usepackage{etoolbox}
\usepackage{remreset}% tiny package containing just the \@removefromreset command
\makeatletter

\usepackage{xcoffins}
\NewCoffin\tablecoffin
\NewDocumentCommand\Vcentre{m}
  {%
    \SetHorizontalCoffin\tablecoffin{#1}%
    \TypesetCoffin\tablecoffin[l,vc]%
  }

\@removefromreset{subsection}{section}
\makeatother
\setcounter{subsection}{1}
\useoutertheme
%\usesectionheadtemplate{\insertsectionhead\hfill}{\color{fg!50!bg}\insertsectionhead\hfill}
%\useoutertheme{split}

\hyphenation{op-tical net-works semi-conduc-tor}
\usepackage{graphicx} 
\usepackage{tikz}

\usepackage{pgfpages}

%% Important 
%%% show note or disable note. 
%\setbeameroption{show notes on second screen=right} % Both

\setbeamertemplate{note page}{\pagecolor{gray!5}\insertnote}\usepackage{palatino}

%%%%%%%%%%%%%%%%%%%%%%%%% code for fading text with gradient coloring table/figure 0 %%%%%%%%%%%%%%%%%%%%%%%%%%%%%%%%%%%
\usepackage{graphicx}
\usetikzlibrary{fadings}
\newcommand\fadingtext[3][]{%
  \begin{tikzfadingfrompicture}[name=fading letter]
    \node[text=transparent!0,inner xsep=0pt,outer xsep=0pt,#1] {#3};
  \end{tikzfadingfrompicture}%
  \begin{tikzpicture}[baseline=(textnode.base)]
    \node[inner sep=1pt,outer sep=1pt,#1](textnode){\phantom{#3}}; 
    \shade[path fading=fading letter,#2,fit fading=false]
    (textnode.south west) rectangle (textnode.north east);% 
  \end{tikzpicture}% 
}
\newcommand*{\TakeFourierOrnament}[1]{{%
\fontencoding{U}\fontfamily{futs}\selectfont\char#1}}
\newcommand*{\danger}{\TakeFourierOrnament{66}}

%\setbeamertemplate{footline}{}
%\setbeamertemplate{footline}{}
\usepackage{xcolor}
\usepackage{soul}

\usepackage{etoolbox}
\makeatletter
%\patchcmd{\slideentry}{\ifnum#2>0}{\ifnum2>0}{}{\@error{unable to patch}}% replace the subsection number test with a test that always returns true
\makeatother
%\DeclareRobustCommand{\hlcyan}[1]{{\sethlcolor{blue}\hl{#1}}}
%\useoutertheme{miniframes}
%\AtBeginSection[]{\subsection{}}
%\definecolor{babyblue}{rgb}{0.54, 0.81, 0.94}
%\usepackage{MnSymbol,wasysym}
% Change 
%\setbeamercolor{alerted text}{fg=cvut_navy}
% color the header UPC colors 
\setbeamercolor*{palette primary}{bg=cvut_navy,fg=gray!20!white}
\setbeamercolor*{palette secondary}{bg=cvut_navy,fg=gray!20!white} % no color
%\setbeamercolor*{palette secondary}{bg=cvut_navy,fg=cvut_navy}
%\setbeamercolor*{palette secondary}{bg=cvut_blue,fg=white}
\setbeamercolor*{palette tertiary}{parent=palette primary} % color of the top and date
\setbeamercolor*{palette quaternary}{fg=cvut_navy,bg=gray!5!white}
\setbeamercolor*{sidebar}{fg=cvut_navy,bg=gray!15!white}
\usepackage[first=0,last=9]{lcg}
\newcommand{\ra}{\rand0.\arabic{rand}}
\usepackage{color, colortbl}
\usepackage{stackengine,tikz}
\usepackage{transparent}
\usepackage{pgfpages}
\usepackage{graphicx}% http://ctan.org/pkg/graphicx
\usepackage{booktabs}% http://ctan.org/pkg/booktabs
%\setbeameroption{show notes}
\colorlet{Gray}{gray!30}
\newcommand{\bert}{\ensuremath{%
  \mathchoice{\includegraphics[height=2ex]{Bert-pic-removebg-preview.png}}
    {\includegraphics[height=2ex]{Bert-pic-removebg-preview.png}}
    {\includegraphics[height=1.5ex]{Bert-pic-removebg-preview.png}}
    {\includegraphics[height=1ex]{Bert-pic-removebg-preview.png}}
}}


%\colorlet{Gray}{gray!30}
\newcommand{\trad}{\ensuremath{%
  \mathchoice{\includegraphics[height=2ex]{trad_off_acc.pdf}}
    {\includegraphics[height=2ex]{trad_off_acc.pdf}}
    %{\includegraphics[height=1.5ex]{trad-off.pdf}}
    {\includegraphics[height=1.5ex]{trad_off_acc.pdf}}
    {\includegraphics[height=1ex]{trad_off_acc.pdf}}
}}


\newcommand{\tradd}{\ensuremath{%
  \mathchoice{\includegraphics[height=2ex]{trad_simple.pdf}}
    {\includegraphics[height=2ex]{trad_simple.pdf}}
    %{\includegraphics[height=1.5ex]{trad-off.pdf}}
    {\includegraphics[height=1.5ex]{trad_simple.pdf}}
    {\includegraphics[height=1ex]{trad_simple.pdf}}
}}



\newcommand{\tradda}{\ensuremath{%
  \mathchoice{\includegraphics[height=2ex]{dataset-trad.pdf}}
    {\includegraphics[height=2ex]{dataset-trad.pdf}}
    %{\includegraphics[height=1.5ex]{trad-off.pdf}}
    {\includegraphics[height=1.5ex]{dataset-trad.pdf}}
    {\includegraphics[height=1ex]{dataset-trad.pdf}}
}}

\newcommand{\tradall}{\ensuremath{%
  \mathchoice{\includegraphics[height=2ex]{ALL-trad.pdf}}
    {\includegraphics[height=2ex]{ALL-trad.pdf}}
    %{\includegraphics[height=1.5ex]{trad-off.pdf}}
    {\includegraphics[height=1.5ex]{ALL-trad.pdf}}
    {\includegraphics[height=1ex]{ALL-trad.pdf}}
}}


\newcommand{\tradf}{\ensuremath{%
  \mathchoice{\includegraphics[height=2ex]{fast-accur.pdf}}
    {\includegraphics[height=2ex]{fast-accur.pdf}}
    %{\includegraphics[height=1.5ex]{trad-off.pdf}}
    {\includegraphics[height=1.5ex]{fast-accur.pdf}}
    {\includegraphics[height=1ex]{fast-accur.pdf}}
}}

\newcommand{\tradsf}{\ensuremath{%
  \mathchoice{\includegraphics[height=2ex]{fast-simple.pdf}}
    {\includegraphics[height=2ex]{fast-simple.pdf}}
    %{\includegraphics[height=1.5ex]{trad-off.pdf}}
    {\includegraphics[height=1.5ex]{fast-simple.pdf}}
    {\includegraphics[height=1ex]{fast-simple.pdf}}
}}

\newcommand{\theauthor}[1]{%
\includegraphics[width=0.5\textwidth]{#1}\\#1
}%




%\definecolor{studentbrown}{RGB}{124,71,50}
%\definecolor{studentbrown}{RGB}{124,71,50}

%\definecolor{studentbrown}{rgb}{0.87, 0.19, 0.39} % pink block

%\definecolor{studentbrown}{rgb}{0.82, 0.1, 0.26}


\newcommand*{\MinNumber}{0}%
%\newcommand*{\MaxNumber}{100}%
\newcommand*{\MaxNumber}{0.4}%
\definecolor{bubblegum}{rgb}{0.99, 0.76, 0.8}
\newcommand{\ApplyGradient}[1]{%
  \pgfmathsetmacro{\PercentColor}{100.0*(#1-\MinNumber)/(\MaxNumber-\MinNumber)}%
  %\textcolor{black!\PercentColor}{#1}
  \edef\x{\noexpand\cellcolor{babyblue!\PercentColor}}\x\textcolor{black}{#1}%
}
\newcolumntype{R}{>{\collectcell\ApplyGradient}{c}<{\endcollectcell}}

%\setbeameroption{show notes on second screen=right}
\setbeamercolor{titlelike}{parent=palette primary}
\setbeamercolor{frametitle}{parent=palette primary}

\setbeamercolor{B}{bg=red!30,fg=black}

 %\setbeamertemplate{itemize item}{\color{cvut_bbredlue}$\blacksquare$}
%\setbeamertemplate{itemize item}{\color{cvut_blue}$circle$}
%\setbeamercolor{item projected}{bg=magenta!70!black,fg=white}
 %\setbeamertemplate{itemize item}[circle]

\setbeamertemplate{section in toc}[default]
\setbeamercolor{itemize item }{fg=blue}
\setbeamertemplate{itemize item}[circle]

\setbeamercolor*{separation line}{}
\setbeamercolor*{fine separation line}{}

\setbeamertemplate{navigation symbols}{} 
\setbeamertemplate{caption}{\raggedright\insertcaption\par}

%\setbeamercolor*{block title example}{fg=blue!50,bg= blue!10}

%\setbeamercolor*{block title example}{fg=white,bg= cvut_navy}
\setbeamercolor*{block title example}{fg=white,bg=purple!75!black}
\setbeamercolor*{block body example}{fg= black, bg= white}

%fg=black,bg=purple!75!black
%\setbeamercolor*{block body example}{fg= blue, bg= blue!5}
%\setbeamercolor*{block body example}{fg= white, bg= cvut_blue}
%diffrent color 
%bg=magenta!70!black,fg=white

\setbeamercolor{itemize item}{fg=cvut_navy} % all frames will have red bullets
\setbeamercolor{block title}{bg=red!30,fg=black}
\setbeamertemplate{subsection in toc}[subsections numbered]

\usepackage{eqnarray,amsmath}
\usepackage{amsfonts}
\usepackage{amssymb}
\usepackage{qrcode}
\usepackage{graphicx}
\usepackage{lmodern} % pro pismo tucne a zaroven kurziva
\usepackage{bm} % pro pismo tucne a zaroven kurziva
\usepackage{epstopdf}
\usepackage{changepage}
%\setbeamercolor{structure}{fg=darkred}

%\setbeamercolor{block title}{fg=darkred,bg=structure.fg!10!bg!10!bg}
%\setbeamercolor{block body}{use=block title,bg=block title.bg}
\definecolor{NormalBlue}{RGB}{200,200,255}
%\setbeamercolor{block title}{bg=NormalBlue}yellow
\setbeamercolor{block title}{fg=black, bg=yellow}
\setbeamercolor{block2}{use=structure,fg=white,bg=purple!75!black}
% definice makra
\def\bq{\mbox{\kern.1ex\protect\raisebox{-1.3ex}[0pt][0pt]{''}\kern-.1ex}}
\def\eq{\mbox{\kern-.1ex``\kern.1ex}}
\def\ifundefined#1{\expandafter\ifx\csname#1\endcsname\relax }%
\ifundefined{uv}%
        \gdef\uv#1{\bq #1\eq}
\fi


%====================================================
%========== DEFINITION OF AUTHORS ETC...=============
%====================================================
\author[PhD Thesis Defense]{ Your Name}
\institute[]{Department of Computer Science \\ TALP Group 
\vspace{2mm} \\ 
Advisors:  \\ Prof. John \\
Dr. Rosa\\ 
\vspace{2mm}}
\title[XX ]{your title}


  \date[Feb 10, 2021]{PhD Thesis Defense \\ \small{Feb 10, 2021}}
%====================================================
%========== BEGINNING OF DOCUMENT ===================
%====================================================


\begin{document}



\begin{frame}[plain]
%\transboxin

	\titlepage
	\begin{center}%
	%\vspace{-2mm}
	%\tableofcontents{current}
  		\includegraphics[height=1cm]{darkblue-logo.pdf}
  		%\includegraphics[height=1cm]{darkblue-talp-logo.pdf}
  		%\includegraphics[height=1.5cm]{upclogo.pdf}
		%\includegraphics[height=1.1cm]{files/IRIS.png}
		%\includegraphics[height=1.3cm]{files/logo_iri2.png}https://www.overleaf.com/project/5bed4779c7eea31291e7e8b2#Navigation27
	%\includegraphics[height=1.3cm]{files/IBT.png}
		%\includegraphics[height=1.3cm]{files/biocev-logo-CMYK-horizontal.pdf}
	\end{center}
\note[item]{you  note here}
\end{frame}


%\logo{\includegraphics[height=1cm]{files/TALP.png}}
%\logo{\includegraphics[height=0.6cm]{darkblue-logo.pdf}}
%\logo{\includegraphics[height=0.5cm]{darkblue-logo.pdf}}

%% outline index   
 % \begin{frame} 
 % \frametitle{Outline}
 % \tableofcontents[currentsection]
 % \end{frame}
%}


\begin{frame}[plain]{Table of Contents} 
     \tableofcontents[currentsection]
\end{frame}


\begin{frame}[plain]{Table of Contents}
\tableofcontents[]
\note[item]{note here }
\end{frame}



%\section{Introduction}


\section{Problem Identification}

\begin{frame}

	\frametitle{Highlight with a block}	
	
%\begin{block}{Text Recognition}
\begin{block}{\samll}<0>

\begin{itemize}
\item [A] Fixed lexicon [\textcolor{darkblue}{some text }].
%\item Lexicon free [\textcolor{darkblue}{shi et al. 2016, Ghosh et al. 2017, Gao et al. 2017, Fang et al. 2018, Xing et al., 2019, Hu et al. 2020}].
	%\item \textbf{Text}
    %\item \textcolor{cvut_navy}{Text} 
    %\item \textcolor{cvut_navy}{\textbf{Text}} 
\end{itemize}

\end{block}

\begin{block}{\samll}<0>

\begin{itemize}
%\item Fixed lexicon [\textcolor{darkblue}{Wang et al.  2012, Jaderberg et al. 2016}].
\item [B]  Lexicon free [\textcolor{darkblue}{some text }].
	%\item \textbf{Text}
    %\item \textcolor{cvut_navy}{Text} 
    %\item \textcolor{cvut_navy}{\textbf{Text}} 
\end{itemize}

\end{block}
%\includegraphics[width=0.8\textwidth]{intro.pdf}
%\includegraphics[width=0.7\textwidth \textcolor{white}{right}]{accv-overview-2.pdf}
\uncover<1>{
}
\vspace{-0.1cm}
%\tiny Cnnectionist Temporal Classification CTC [\textcolor{darkblue}{Graves et al. 2006}]


\end{frame}


\begin{frame}

	\frametitle{Highlight with a block}	
	
%\begin{block}{Text Recognition}
\begin{block}{\samll}

\begin{itemize}
\item [A] Fixed lexicon [\textcolor{darkblue}{some text}].
%\item Lexicon free [\textcolor{darkblue}{shi et al. 2016, Ghosh et al. 2017, Gao et al. 2017, Fang et al. 2018, Xing et al., 2019, Hu et al. 2020}].
	%\item \textbf{Text}
    %\item \textcolor{cvut_navy}{Text} 
    %\item \textcolor{cvut_navy}{\textbf{Text}} 
\end{itemize}

\end{block}

\begin{block}{\samll}<0>

\begin{itemize}
%\item Fixed lexicon [\textcolor{darkblue}{Wang et al.  2012, Jaderberg et al. 2016}].
\item [B] Lexicon free [\textcolor{darkblue}{some text}].
	%\item \textbf{Text}
    %\item \textcolor{cvut_navy}{Text} 
    %\item \textcolor{cvut_navy}{\textbf{Text}} 
\end{itemize}

\end{block}
%\includegraphics[width=0.8\textwidth]{intro.pdf}
%\includegraphics[width=0.7\textwidth \textcolor{white}{right}]{accv-overview-2.pdf}

\vspace{0.1cm}


\end{frame}


 \begin{frame}[noframenumbering]{Add Block in order A  ...}
  \begin{alertblock}{\small}

\begin{itemize}
    \item [A] \textbf{Lack of public dataset:} Most state-of-art deep ...... 
%\item \textbf{Fast and easy to re-train:} Statistical Language Modelling (LM) can be trained on specific domain.    
%\item The system can be used as a \textbf{drop-in replacement} for any text-spotting algorithm that ranks the output words.
%\item This \textbf{hybrid approach} between deep learning and classical statistical modelling opens the possibility to produce accurate results with very simple models. 
\end{itemize}
\end{alertblock}
\visible<2->{
\begin{alertblock}{\small}
\begin{itemize}
 %   \item \textbf{Lack of public dataset:} Most state-of-art deep models trained on synthetic dataset. 
\item [B] \textbf{Fast and easy to re-train:} Statistical Language  ....   
%\item The system can be used as a \textbf{drop-in replacement} for any text-spotting algorithm that ranks the output words.
%\item This \textbf{hybrid approach} between deep learning and classical statistical modelling opens the possibility to produce accurate results with very simple models. 
\end{itemize}
\end{alertblock}
}
\visible<3->{
\begin{alertblock}{\small}
\begin{itemize}
  %  \item \textbf{Lack of public dataset:} Most state-of-art deep models trained on synthetic dataset. 
%\item \textbf{Fast and easy to re-train:} Statistical Language Modelling (LM) can be trained on specific domain.    
\item [C] The system can be used as a  ..... 
%\item This \textbf{hybrid approach} between deep learning and classical statistical modelling opens the possibility to produce accurate results with very simple models. 
\end{itemize}
\end{alertblock}
}

\visible<4->{
\begin{alertblock}{\small}
\begin{itemize}
  %  \item \textbf{Lack of public dataset:} Most state-of-art deep models trained on synthetic dataset. 
%\item \textbf{Fast and easy to re-train:} Statistical Language Modelling (LM) can be trained on specific domain.    
%\item The system can be used as a \textbf{drop-in replacement} for any text-spotting algorithm that ranks the output words.
\item [D] This \textbf{hybrid approach} between deep learning  ..... 
\end{itemize}
\end{alertblock}
}





\note[item]{\textbf{Block adding ...} \\ as I mentioned before  \\ 

\textcolor{blue}{(A-)}  all STOA models are trained on ....  \\ 

\textcolor{blue}{(B-)} \textbf{Second}, Fast and easy t...  \\

\\


\textcolor{blue}{(C-)} \textbf{Also}, the model can be used as a \textbf{drop-in replacement} for any text-spotting algorithm that ranks the output words. \\
\\ 

And \\
\\ 
\textcolor{blue}{(D-)} \textbf{finally} combining deep learning with a classical model can achieve a better result with low cost model (even training on CPU).I'm not generalizing here just in this case}
\end{frame}






\note[item]{\textbf{Why post-processing?} \\ as I mentioned before  \\ 

\textcolor{blue}{(A-)}  all STOA models are trained on this one synthetic dataset. \\ 

\textcolor{blue}{(B-)} \textbf{Second}, Fast and easy to re-train and can be adopted to any specific domain. \\

\\


\textcolor{blue}{(C-)} \textbf{Also}, the model can be used as a \textbf{drop-in replacement} for any text-spotting algorithm that ranks the output words. \\
\\ 

And \\
\\ 
\textcolor{blue}{(D-)} \textbf{finally} combining deep learning with a classical model can achieve a better result with low cost model (even training on CPU).I'm not generalizing here just in this case}
\end{frame}





\begin{frame}{Literature review block}
\frametitle{Literature review block}
%\begin{center}
Work addresses scene understanding, and benefit from combining text cue and visual context in image or text retrieval:  


\\~~\\


\vspace{-1cm} 

\begin{minipage}{4cm}
\begin{alertblock}{\small Lexicon Generation}
 \textcolor{darkblue}{Patel et al. (2016)}
\end{alertblock}
\end{minipage}
\space \space \space generation of new lexicon with \textbf{topic modeling}  




\begin{minipage}{4cm}
\begin{block}{\small Logo Retrieval}
 \textcolor{darkblue}{Karaoglu et al. (2017)}
\end{block}
\end{minipage}
    \space \space \space learn \textbf{textual information} from logos   \\ 


\begin{minipage}{4cm}
\begin{alertblock}{\small Text detection}
 \textcolor{darkblue}{Prasad et al. (2018)}
\end{alertblock}
\end{minipage} \space \space \space using object information for text detection




%\end{center} 
 \note[item]{show this slide 2 seconds }        
\end{frame}



\begin{frame}[noframenumbering]{Literature review block }
%\section{literature Review}
\setbeamercovered{transparent} 
%\begin{center}
%Work addresses scene understanding, text cue or visual context in text spotting:  
%Work addresses scene understanding, and benefit from combining text cue and visual context in \textcolor{red}{text} retrieval: 
Work addresses scene understanding, and benefit from combining text cue and visual context in image or text retrieval:  


\\~~\\


\vspace{-1cm} 
\begin{minipage}{4cm}
\begin{alertblock}{\small Lexicon Generation}
 \textcolor{darkblue}{Patel et al. (2016)} 
\end{alertblock}
\end{minipage}
\space \space \space generation of new lexicon with \textbf{topic modeling}  


\begin{minipage}{4cm}
\begin{block}{\small Logo Retrieval}<0>
 \textcolor{darkblue}{Karaoglu et al. (2017)}
\end{block}
\end{minipage}
    \space \space \space \texttransparent{0.1}{learn \textbf{textual information} from logos}   \\ 


 
\begin{minipage}{4cm}
\begin{alertblock}{\small Text detection}<0>
 \textcolor{darkblue}{Prasad et al. (2018)}
\end{alertblock}
\end{minipage} \space \space \space  \texttransparent{0.1}{using object information for text detection}

 
\note[item]{I will start with \textbf{topic modeling and lexicon generations} }  
 
 \end{frame}


\begin{frame}[noframenumbering]
	\frametitle{Image}	
\begin{itemize}

\item The task involves detecting the viewer’s interpretation of an Ad image captured as text.
\item Fine-tune BERT  is used to learn textual and visual cues. 
%\item Specific domain dataset is used:  Ads Dataset
\item Google Vision API is used to extract scene text information.
\end{itemize}

\begin{figure}
\begin{center}
   
%%%\vspace{10pt}
%\vspace{-0.3cm}
\vspace{0.2cm}   
\includegraphics[width=0.6\textwidth]{BERT-2020.pdf}

    
    \label{ActionButton}
  \end{center}
\end{figure}
\note[item]{}   
\end{frame}



\begin{frame}
	\frametitle{Table}	
\begin{itemize}
%\item Unigram Language Model (ULM)

\item The ULM is based on a combined corpu .... 

%\item The advantage of ULM is very simple to build, train and \textcolor{red}{adapt to new domains} opening the possibility to improve baseline performance for specific applications. 
 

\end{itemize}
\begin{block}{}
\begin{table}
\centering
%\caption{Total count of unique words - Dictionary.} 
\resizebox{\columnwidth}{!}{
\begin{tabular}{|l|c|c|c|c|}

%\begin{tabularx}{\columnwidth}{|lcccc|}
\hline 
 
\multicolumn{5}{|c|}{Unique Count of Textual Data}\\ 
\hline 
\rowcolor{Gray}
 %INPUT &  \cellcolor{applegreen}{token} & \cellcolor{red}{$\leq$ PAD $\geq$} & \cellcolor{amber}{300D}  \\
 Dictionary & words  & nouns & verb & adjectives                        \\
 \hline 
 %\rowcolor{Gray}
%\multicolumn{5}{|c|}{Dictionary} \\ 
% \hline 
Dict-90K words level       &   87,629     &   20,146     &  6,956    & 15,534        \\   %\hdashline
Language model     & 8870,209 & 2695,906 & 139,385 &  824,581  \\  %\hdashline 
  \hline 


\end{tabular}
}
%\end{tabularx}
%\label{se:Dic}
\end{table}
\end{block}
\note[item]{.}
\end{frame}

\begin{frame}[noframenumbering]{Fade out  text}
%\begin{itemize}
   % \item In this thesis, \hl{ we have demonstrated the benefit of leveraging these NLP techniques into computer vision problems that boost the performance of the text spotting system up to 3.3 points on a benchmark dataset without tuning or training}. Therefore, to answer to our main question:
%\end{itemize}

\begin{block}{\small}

%The work in the present dissertation has addressed this specific problem.
 \begin{itemize}
     \item The main limitation of this approach is that \textbf{depends on the baseline softmax} output to re-rank the most closely related word
     %\item In particular, the \hl{semantic relatedness score suppresses unrelated words and boosts the most probably related word} by simple dot product multiplication.
 \end{itemize}

%\end{itemize}
\end{block}    

\begin{block}{\small}

%The work in the present dissertation has addressed this specific problem.
 \begin{itemize}
    % \item The main limitation of this approach is that \textbf{depends on the baseline softmax} output to re-rank the most closely related word
     \item [] As in the example, the semantic relatedness score \textbf{\textcolor{gray}{suppresses unrelated words}} and \textbf{\textcolor{blue}{boosts the most probably related word}} by simple dot product multiplication. (Visual context: parking meters)
 \end{itemize}

%\end{itemize}
\end{block}    

\begin{table}[t]
\begin{center}
%\caption{Sample of the modified probabilities before and after the visual context, As shown in Figure \ref{fig:overview-of-all-model} these individual word scores are modified (re-ranked) based on the presence of others object classes, scene classes and image caption which are extracted through state-of-the-art frameworks from each problem domain.} 
%Sample of the modified probabilities before and after the visual
%\caption{Sample of the modified probabilities before and after the visual context. As shown these individual word scores are modified (re-ranked) based on the presence of other object classes and scene classes which are extracted through state-of-the-art frameworks from each problem domain.} 
\begin{threeparttable}
%\begin{tabular}{l|rl|rl|rl|rl}
%\begin{tabular}{l|rl|rl|rl}
%\hline \bf \small word & \ \small w1   & \small p1     & \small w3  &  \small p3 \\ \hline
\begin{tabular}{|c|cl|cl|}
%\hline \bf \small word & \ \small w1   & \small p1     & \small w3  &  \small p3 \\ 
\hline 
 \bf $w$ &  \multicolumn{2}{c|} {\small Text Spotting Model} & \multicolumn{2}{c|}{ \small Visual Re-ranker Model}  \\
\hline \hline 
%$w1$ & \fadingtext{top color=gray,bottom color=gray}{quotas}      &  $0.5$ & \bf quarters &  $5.4\text{e-}7$  \\
$w_{1}$ & quotas      &  $0.5$ & \fadingtext{top color=blue,bottom color=blue}{\textbf{quarters}} &  $5.4\text{e-}7$  \\
%$w_{1}$ & quotas      &  $0.5$ & \fadingtext{top color=black,bottom color=back}{\textbf{quarters}} &  $5.4\text{e-}7$  \\
%$w_{1}$ & quotas      &  $0.5$ & \textcolor{blue}{\textbf{quarters}} &  $5.4\text{e-}7$  \\
 $w_{2}$ & quartos     &  $0.1$ &  \fadingtext{top color=gray,bottom color=gray}{quartos} &   $5.2\text{e-}8$  \\ 
% $w2$ & quartos     &  $0.1$ & quartos &   $5.2\text{e-}8$  \\ 
$w_{3}$ & \bf quarters &  $0.05$ & \hbox{ \fadingtext{top color=gray,bottom color=gray}{\textcolor{red}{quotas}}} &   $9.0\text{e-}9$ \\ 



\hline\end{tabular}
%\label{table_1}
%\begin{tablenotes}
 %     \small 
      %\item  \textbf{Bold font} word indicate ground truth.
      %\item Visual information 
      %*VCI (visual context information), result of best three word, see Figure \ref{figure1} for more details. %However, we overcome the limitation of false recognition of short words of the baseline.    
%\end{tablenotes}

\end{threeparttable}
\label{tb: modified prob-ch4} 
\end{center}
%\caption{fddddddddddddddd }

\end{table}

\end{frame}




\begin{frame}[plain]{Math block}
\begin{block}{Attention 1}
\begin{equation*}
c_{t}=\sum_{j=1}^{T} \alpha_{t j} h_{j}, \alpha_{t j}=\frac{\exp \left(e_{t j}\right)}{\sum_{k=1}^{T} \exp \left(e_{t k}\right)}, e_{t j}=a\left(s_{t-1}, h_{j}\right)
\end{equation*}


\end{block}

%The attention mechanisms provide the model with  direct access between state at a different point in time. \cite{bahdanau2014neural} introduces an attention model that computes the context vector $c_{t}$ as the weighted mean of $h$ state sequence, given the model the hidden state $h_{t}$ at each time step:



%\noindent where $\alpha_{ti}$ is the weight computed at each time $t$ step for hidden state $h_{j}$, %T is the number of time steps for the input sequence. The $c$ context vector is used to compute the new state sequence $s$, where $s_{t}$ depends on previous state $s_{t-1}$. The  $\alpha_{t j}$ weight are then computed as: 

\begin{block}{Attention 2}
\begin{equation*}
c_{t}=\sum_{j=1}^{T} \alpha_{t j} h_{j}, \alpha_{t j}=\frac{\exp \left(e_{t j}\right)}{\sum_{k=1}^{T} \exp \left(e_{t k}\right)}, e_{t j}=a\left(\cancel{\textcolor{red}{s_{t-1}}}, h_{j}\right)
\end{equation*}


\end{block}


\end{frame}

\begin{frame}{Figure}
\centering 
\begin{tikzpicture}
    \begin{axis}[
    %legend pos=south east,
    legend pos=north east,
    %legend pos=outer north east
grid=major,
       % legend
%yticklabel=\empty,
%legend pos=outer north east,
legend cell align={left}, % all 
        grid,
        %legend style={fill=none},
        %xlabel=\textsc{Re-ranked caption},
        xlabel=Caption,
        ylabel=  BLEU, 
       % legend style={
        %    at={(0.5,0.96)},
        %    anchor=west,
        %    mark size=1pt,
        %    legend columns=-1,
           % /tikz/every even column/.append style={column sep=0.cm}
         %   every node near coord/.append style={font=\tiny}Sim
        %},
    ]

%    \axispath\draw
   %         (7.49165,-10.02171)
%        |-  (8.31801,-11.32467)
%        node[near start,left] {$\frac{dy}{dx} = -1.58$};

%\addplot [blue!30, smooth,mark=*, line width=0.5mm] coordinates {
\addplot [only marks,
    scatter,
    mark=halfcircle*,
   %mark=triangle*,
    mark options={rotate=90},
    mark size=2.9pt] coordinates {
%\addplot [scatter, mark=*, mark size=3pt, line width=2pt, mesh] coordinates {
      %\addplot [draw=none, fill=pistachio!40] coordinates {
% B1 beam
(1, 0.54)
(2, 0.29)
(3, 0.31)
(4, 0.43)
(5, 0.42)
(6, 0.39)
(7, 0.55)
(8, 0.26)
(9, 0.33)
(10,0.33)
(11,0.44)
(12,0.44)
(13,0.33)
(13,0.33)
(14, 0.44)
(15, 0.58)
(16,0.19)
(17, 0.18)
(18, 0)
(19, 0.43)
(20, 0.35)
(21, 0.29)
(22, 0.26)
(23, 0)
(24, 0.20)
(25, 0)
(26,0.23)
(27, 0.21)
(28, 0.21)
(29, 0.20)
(30, 0.23)
%(31, 0.44)
%(32, 0) 
%(1, 0.88)
%(2, 0.88)
%(3, 0.89)
%(4, 0.89)
%(5, 0.9)
%(6, 0.94)
%(7, 0.94)
%(8, 0.88)
%(9, 0.88)
%(10, 0.94)
%(11, 0.93)
%(12, 0.91)
%(13, 0.89)
%(14, 0.88)
%(15, 0.93)
%(16, 0.97)
%(17, 0.83)
%(18. 0.88)
%(19,0.88)
    };
%darksalmon!50
%\addplot [darksalmon!90,  smooth, mark=square*, line width=0.5mm]  coordinates {
\addplot [only marks,
    scatter,
    mark=halfcircle*,
    mark size=2.9pt] coordinates {

(1, 0.53)
(2, 0.49)
(3, 0.41)
(4, 0.90)
(5, 0.54)
(6, 0.37)
(7, 0.66)
(8, 0.17)
(9, 0.30)
(10,0.44)
(11, 0.33)
(12, 0.54)
(13, 0.22)
(14, 0.44)
(15, 0.49)
(16, 0.19)
(17, 0.32)
(18, 0.40)
(19, 0.19)
(20, 0.85)
(21, 0.40)
(22, 0.28)
(23, 0.40)
(24, 0)
(25, 0.25)
(26, 0)
(27, 0)
(28, 0.51)
(29,0)
(30, 0.23)
%(31, 0.33)
%(32, 0)


  %  \addplot plot coordinates {
%   SBERT-sts


%(1, 0.27)
%(2, 0.58)
%(3, 0.03)
%(4, 0.76)
%(5, 0.73)
%(6, 0.57)
%(7, 0.98)
%(8, 0.80)
%(9, 0.80)
%(10, 0.44)
%(11, 0.78)
%(12, 0.57)
%(13, 0.69)
%(14, 0.69)
%(15, 0.75)
%(16, 0.47)
%(17, 0.81)
%%(18, 0.90)
%%(19, 0.20)
    };
    
        \addplot [black!30,  smooth,  line width=0.8mm] coordinates {
%    %\addplot [blue(ncs)!20, line width=1mm] coordinates {
%    (1,0.40)
%(2, 0.18)
%(3, 0)
%(4, 0.43)
%(5, 0.35)
%(6, 0.29)
%(7, 0.26)
%(8, 0)
%(9, 0.20)
%(10, 0)
%(11,0.23)
%(11, 0.21)
%(12, 0.21)
%(13, 0.20)
%(14, 0.23)
%(15, 0.44)
%(16, 0) 
%(1, 0.53)
%(2, 0.49)
%(3, 0.37)
%(4, 0.66)
%(5, 0.44)
%(6, 0.54)
%(7,0.22)
%(8, 0.17)
%(9, 0.33) %
%(10,0.44)
%(11, 0.19)
%(12, 0.54)
%(13,0.30)
%(14, 0.90) 
%(15, 0.49) 
%(16, 0.53) 
%(17, 0.45)


};




%\addplot [green!20, mark=+,  line width=0.8mm]  coordinates {

%(1,0.26)
%(2, 0.23)
%(3, 0.15)
%(4, 0.95)
%(6, 0.25)
%(7, 0.25)
%(8, 0.24)
%(9, 0.10)
%(10, 0.17)
%(11, 0.12)
%(12, 0.18)
%(13, 0.31)
%(14,0.18)
%(15, 0.25)
%(16, 0.25)
%};

    %\addplot plot coordinates {
\addplot +[darksalmon!50,  line width=0.8mm] coordinates {
%    \addplot [blue(ncs)!20, line width=1mm] coordinates {
    
%(1, 0.32)
%(2, 0.40)
%(3, 0.19)
%(4, 0.85)
%(5, 0.40)
%(6, 0.28)
%(7, 0.40)
%(8, 0)
%(9, 0.25)
%(10, 0)
%(11, 0)
%(12, 0.51)
%(13,0)
%(14, 0.23)
%(15, 0.33)
%(16, 0)
%(1, 0.40)
%(2, 0.33)
%(3, 0)
%(4, 0.16)
%(5, 0.85)
%(6, 0.50)
%(7, 0.667)
%(8, 0.667)
%(9, 0.99)
%(10, 0.87)
%(11, 0.50)
%(12, 0.25)
%(13, 0.12)
%(14, 0.99)
%(15, 0.62)
%(16, 0.33)
%(17, 0.99)
%(18, 0.87)
%(19, 0.99)
    };
    
   % BLEU

\legend{ \small Baseline \\ Similarity\\}
    \end{axis}
\end{tikzpicture}



\end{frame}
\centering 
\begin{frame}{Figure 2}
\centerning 
\begin{tikzpicture}
  \centering
  \begin{axis}[
        ybar, axis on top,
        title={},
        legend cell align={left},
        %grid=both,
        grid=minor,
        %xmajorgrids=true,
        %ymajorgrids=true,
        %grid=major,
        %minor tick num=5,
       % width=\textwidth,
        %height=8cm, width=10.5cm,
        height=6cm, width=8.5cm,
        %axis x line*=bottom,
        %axis y line*=bottom,
        ymin=0,
        %legend cell align={left},
        %enlarge x limits={abs=0.5},
      %  ybar,
      % enlargelimits=-0.15,
        %enlarge x limits=0.25,
       % enlarge y limits={upper,value=0.2},
        %legend cell align=left,
        %enlarge x limits=0.5,
        %major tick length=0cm,
        %height=8cm, width=6.5cm,
        bar width=0.14cm,
        %ymajorgrids, tick align=inside,
        %grid=major,
        ymajorgrids, tick align=inside,
        major grid style={draw=white}, 
        %enlarge y limits={value=.001,upper},
        ymin=1000, ymax=13000,
       % ybar, enlarge x limits={abs=1}
        enlarge x limits={abs=0.1cm}
        axis x line*=bottom,
        max space between ticks=14pt,
        %max space between ticks=40pt,
        %axis y line*=right,
        %enlargelimits=0.15,
        %ymajorgrids, tick align=inside, % put more numbers
        %axis y line*=left,
        %axis x line*=left,
        %xmin=0
      % axis y line*=left,
        %enlarge x limits={abs=0.5cm}
        %enlarge x limits={abs=1}
        %enlarge y limits={abs=1}
        %ymin=0
        %xticklabel style={rotate=90,anchor=base,yshift=-0.4cm,xshift=-0.9cm,color=black},
        % xticklabel style={rotate=90,yshift=-0.2cm,xshift=0,color=black},
           xticklabel style={rotate=90,yshift=-0.05cm,xshift=0,color=black},
         % texxt inside the bar
         %xticklabel style={rotate=90,yshift=-0.4cm,xshift=2cm,color=black},
        tickwidth=0pt,
        enlarge x limits=true,
        legend cell align={right}
        %title=Left-align Legend with pgfplots,
        %legend cell align={left}
        %legend style={draw=none}
        legend style={
            at={(0.5,0.96)},
            anchor=west,
            mark size=1pt,
            legend columns=-1,
           % /tikz/every even column/.append style={column sep=0.cm}
            every node near coord/.append style={font=\tiny}
        },
        ylabel={Frequency},
        symbolic x coords={
           %Baseline ,$\bert$ BERT,$\diamondsuit$ FDCLSTM, $\spadesuit$ TWE, 
     ski, clothing, ballplayer, tree, pizza, street, sign, car, washbasin, table, plate, human, light, seat, airliner, baseball
           %Jan-12,Feb-12,
           %Mar-12,
          },
       xtick=data,
       %enlarge x limits,
       %enlarge x limits=0.2,
       %enlarge y limits={rel=-0.9,upper},
       enlarge x limits={abs=0.01},
       enlarge x limits=0.05,
        enlarge y limits=0.05,
       %restrict x to domain=1:2
       nodes near coords={
       % \pgfmathprintnumber[precision=0]{\pgfplotspointmeta}
       }
    ]
    %\addplot [draw=none, fill=black!30] coordinates {
    \addplot [draw=none, fill=pistachio!40] coordinates {
      %(Baseline , 26.2)
     %  (,6000 )
      (ski,  12766) 
      (clothing, 9596)  
      (ballplayer, 10747) 
      (tree, 10460)
      (pizza, 8972)
      (street, 8945) 
      (sign, 8659)
      (car, 8080) 
      (washbasin, 8046)  
      (table, 6792)       
      (plate, 6696)       
      (human, 6793)       
      (light, 6335)      
      (seat, 6031) 
      (airliner, 5461)    
      (baseball,  5075)          
      %(plate, 1085)
      %(ballplayer, 14238) 
      %(car, 1069)
     % (racket, 2599)
     % (baseball, 0)
     % (washbasin, 2283)
     %  (elephant, 1223 )
   %   (restaurant, 0)
     % (elephant, 973)
    %  (parachute, 9399)
    %  (traffic, 0)
    %  (umbrella, 9196)
     % (alp, 8868)
     % (passenger,  8353)
     % (pizza, 1209)  
      
      %(#,6000 )
     % (seashore,8181)
      %(wing,8034)
      };
      % };
      %(Jan-12,67.5600) 
      %(Feb-12,88.2339)
      %(Mar-12,78.6138) 
    %  (Apr-12,58.9129) };
   \addplot [draw=none,fill=darksalmon!50] coordinates {
     (ski,  10901)
     (clothing, 6682)    
    (ballplayer, 9468)
    (tree,  8307)
    (pizza, 8546)
    (street, 7690)
    (sign, 7468)
    (car, 6358) 
    (washbasin, 7216)   
    (table, 5456)    
    (plate, 5213)       
    (human, 4917)     
    (light, 5357)   
     (seat, 5714)
     (airliner, 5246) 
     (baseball,  4752)
     %(ball, 1504) 
     %(plate, 4522)
              
     %(pizza,8972)
    % (x, 222) 
     %(baseball, 5153)
     %(car, 3079)
     %(baseball, 2437)
     %(washbasin, 2486)
     %(restaurant, 0)
     %(elephant, 1223 )
    
     %(parachute, 0)
     
     };
    
     % (Jan-12,77.5600) 
     % (Feb-12,78.2339)
     % (Mar-12,88.6138) 
     % (Apr-12,78.9129) }
  % \addplot [draw=none, fill=blue!30] coordinates {
  
   \addplot [draw=none, fill=bananayellow!30] coordinates {
     (ski, 8679)
     (clothing, 3659)    
     (ballplayer,  7844)
     (tree, 5503) 
     (pizza, 7744)       
     (street, 5673)     
     (sign, 5541)
     (car, 4168)     
     (washbasin, 5808)   
     (table, 3728)    
     (plate, 3327)     
     (human, 2898)
     (light, 4014)
     (seat, 5002) 
     (airliner, 4501)    
     (baseball,  4186 )
    %  (x, 222) 
        % (LSTM-V (K3), 40)
      %(LSTM-V (K3), 434)
    % (CNN-90K (CNN-90K (K4)), 345)
     	
      };
    
  \legend{Sim $th\geq 0.2$, Sim $th\geq 0.3$, Sim $th\geq 0.4$ }
  \end{axis}
  \end{tikzpicture}




\end{frame}




 \end{document}
% =============================================================
% =========================== END =============================
% =============================================================

